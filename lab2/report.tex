\documentclass[a4paper,11pt]{article}
%\documentclass[a4paper,9pt,landscape]{article}
\usepackage[english]{babel}
%\usepackage[utf8]{inputenc}
\usepackage{graphicx}
\usepackage{fullpage}
\usepackage{amsmath}
\usepackage{pdfpages}
\usepackage{listings}
\usepackage{color}
\usepackage{multicol}
\usepackage{fancyhdr}
\usepackage[top=1.5cm, bottom=4cm, left=2.5cm, right=2.5cm]{geometry}
\usepackage{hyperref}
\usepackage{verbatim}
\setlength{\voffset}{-9pt}
%\setlength{\hoffset}{-1in}
%\setlength{\marginparsep}{0.5cm}


\setlength{\parindent}{0cm}
%\setlength{\parskip}{-0.1cm}
%\setlength{\columnseprule}{0.4pt}
\setlength{\footskip}{0.5cm}
\setlength{\headheight}{15pt}
\setlength{\headsep}{2cm}

\fancypagestyle{tcr}{%
  \fancyhf{} %clear all headers and footers fields
  \fancyhead[R]{\thepage}
  \fancyhead[L]{\textbf{Assignment No 2 TDTS06}}
  \renewcommand{\headrulewidth}{0.4pt}
}

\definecolor{dkgreen}{rgb}{0,0.6,0}
\definecolor{gray}{rgb}{0.5,0.5,0.5}
\definecolor{mauve}{rgb}{0.58,0,0.82}
\lstset{
  title=\lstname,
  frame=t,
  %aboveskip=-0.5cm, 
  %belowskip=0pt,
  basicstyle=\footnotesize\ttfamily,
  keywordstyle=\color{blue},          % keyword style
  commentstyle=\color{dkgreen},       % comment style
  stringstyle=\color{mauve},         % string literal style
  showstringspaces=false,         % underline spaces within strings
  tabsize=2,
  language=C,
  title=\caption,
  %xleftmargin=-1cm
}


\begin{document}
%% title stuff
\title{Assignment No 2 TDDC78}
\author{Linus Mellberg (linme560)}
\date{\today}
\maketitle
\pagebreak
%\setcounter{page}{1}
%\begin{multicols}{2}
\thispagestyle{tcr}
\pagestyle{tcr}
%\tableofcontents

\section{A simple web proxy with filtering}
\subsection{User manual}
\subsubsection{Building the proxy}
Follow the instructions to build the proxy.
\begin{itemize}
  \item 
    Extract the source code
    \begin{verbatim}$ tar xvjf assignment2_proxy_Linus_Mellberg.tar.bz2
  \end{verbatim}
\item 
  Go to the one of the build directories
  The Debug directory will build the proxy with -g to enable debugging.
  The Proxy directory will build the proxy with -O2 optimize for speed.
  \begin{verbatim}$ cd proxy/Debug\end{verbatim}
  or
  \begin{verbatim}$ cd proxy/Release\end{verbatim}
\item
  Build the project.
  \begin{verbatim}$ make\end{verbatim}
  Make sure you have a c++11 compliant compiler.
  The program isn't tested on the solaris machines, but it should compile on a standard Linux system with a new enough compiler and pthreads support.
  On IDA solaris machines you might have to change the compiler.
  \begin{verbatim}$ module rm prog/gcc
$ module add prog/gcc/4\end{verbatim}
\end{itemize}

\subsubsection{Configuring and running the proxy}
If the build succeded the proxy can be run by
\begin{verbatim}$ ./proxy\end{verbatim}
This will start the proxy listening to port 8080.
Changing the default port can be done by specifing a port number as a parameter.
\begin{verbatim}$ ./proxy 8081\end{verbatim}
The words that the proxy will filter can be configured by editing \emph{filterWords.txt}.
Each uncommented line will be treated as a string to be filtered. Each line can at most be 200 characters long.
Comments can be made by starting a line with \#.
To stop the proxy send a SIGINT signal by pressing ctrl+c, there will be some messages about it stopped accepting connections and exiting.
This means that the main loop has stopped.
If the proxy doesn't exit immediately some threads are still alive and handling connections.
If they don't stop after a while try stopping the browser.
If this doesn't work something has gone wrong and the proxy must be stopped by killing it for example use:
\begin{verbatim}$ killall proxy\end{verbatim}

\subsubsection{Source Code and features}
The program is object oriented and consists of a number of classes. Each is declared and defined in .h and .cc files with corresponding names.
\begin{itemize}
  \item \emph{Connection} is a wrapper for the send and recv calls to a socket.
    It handles buffering of received data. 
    And can return strings of data with a certain length or ending witha  certain byte.
    It can also send strings of data.
  \item \emph{HttpConnection} does some simple HTTP handling. 
    It can parse status lines, headers and then deliver data.
    There is no handling or processing of HTTP here, only parsing.
  \item \emph{Server} handles incoming connections. 
    Setup of listening socket is done with init().
    Each incoming connection must be accepted by calling acceptNew() which will return a Connection object.
  \item \emph{Proxy} handles incoming browser connection.
    When a connection to the browser has been set up a new thread is started. 
    This thread will create a \emph{Proxy} object that will handle this connection until it is closed.
    This is where most of the actual work is done. 
  \item \emph{main.cc} is not an object but runs the basic loop which accepts new connection and dispatches them to new threads.
\end{itemize}

The proxy starts by reading the file \emph{filterWords.txt} to know what to filter.
It configures some handling of POSIX signals to be able to shut down cleanly.
Then it instantiates a Server object which starts listening on the port specified port (or 8080 by default).
The main loop then starts accepting connections.
For each new connection a new thread is started and detached.
Each thread handles the connection by itself and will exit when it is done.
This is done by the \emph{Proxy} object.

The \emph{Proxy} object is started with a \emph{Connection} and a list of words to filter.
It reads the browsers request and starts a connection to the requested server.
The HTTP-request is relayed and the \emph{Connection} header is changed to have the value \emph{Close}.
If the \emph{Accept} header field indicates that the response should contain text, the \emph{Accept-Encoding} header is changed to be empty.
This will stop the web server from compressing the data, making it possible to filter the response.
This migth be bad in terms of bandwidth usage but it makes it possible to filter a lot more pages. 
Many modern web servers compress outgoing data, when this is supported by browsers.

The request is then sent to the webserver.
The response headers are read and again the \emph{Connection} header is changed to the value \emph{Close}. 
This time the \emph{Content-Encoding} and \emph{Content-Type} header fields are checked to see if the response is either compressed or does contain text.
This will decide if filtering will be done.
If filtering will be done, the response data is buffered and filtered.
Otherwise it is read 4096 bytes (or the amount that is received by recv) at a time and sent to the browser.

This is a list of where the required features 2, 3, 6, 7 and 8 are implented.
\begin{itemize}
  \item Feauture 2 is implemented in mostly in \emph{Proxy.cc}, hard to give a single line.
  \item Feauture 3 is implemented in \emph{Proxy.cc} on line 145.
  \item Feauture 6 is implemented, there are no special treatment for any browser. Not testing in IE though.
  \item Feauture 7 is implemented in \emph{main.cc} line 93 and \emph{Server.cc} line 60.
  \item Feauture 8 is implemented in \emph{Proxy.cc} line ~413 checks for what the browser expects back and on line 177 Accept-Encoding is emptied if it is text.
    Incoming data responses are checked at line ~444 and is filtered if Content-Type with text in it is found.
    Filtering is done on line 63.
    Buffering is done on line ~346.
\end{itemize}

\subsection{Functionality and testing}
The proxy has been tested with Chromium and Firefox, both seem to work nicely.
It has been tested on a few pages for example yahoo.com, wikipedia.org, aftonbladet.se and youtube.com.
They all seem to work and are displayed as usual.
Large file transfers has been tested with http://www.thinkbroadband.com/download.html the 1 GB file did download without problems.
Google.com was also tested but this site uses HTTPS which isn't supported by the proxy.
The proxy also works for HTTP connections to non standard ports, this has been tested by downloading the 1 GB test file above through port 81 (there is a link for this on the page).
POST and GET request are supported, any other type of request will generate a HTTP Response with status code 501 Not Implemented.
POST has been tested on http://www.w3schools.com/tags/tryit.asp?filename=tryhtml\_form\_submit the Submit Code button will perform a POST on the form.
IPv6 is not supported.

The proxy was also tested for memory leaks using valgrind, it seems clean.
The filter word list is not freed, but this is intentional and not harmful, only one instance is created.
It should be freed when the last thread exits. 
Doing this would mean going through a lot of trouble to find out when there are no more threads, only to free a pointer right before program exit.

%\clearpage

\end{document}
